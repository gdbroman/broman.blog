\documentclass[12pt]{article}

\usepackage{palatino}
\usepackage{setspace}
\usepackage{titlesec}
\usepackage[super,comma,sort&compress]{natbib}

\title{The Bellian Break}
\author{G.D. Broman}
\date{}

% Double spacing
\doublespacing

% Centering section titles
\titleformat{\section}[block]{\normalfont\Large\bfseries\centering}{\thesection}{1em}{}

\begin{document}
\maketitle

At the end of the 20th century, we took a well-deserved break. The wind-down began physically, with the invention of the atomic bomb, and was concluded virtually, with the booming commercialization of the personal computer. It is common to think of the paralyzing effect these two events have had on megapolitics, but less so what happens to technological momentum when you remove the two most powerful, natural motivators for new ideas: violence and boredom. The break is getting uncomfortably long, and the beginning of the 21st century hints at an empirical answer to the Fermi Paradox; not that civilizations blow themselves up, but that they seduce themselves into retirement. With Soma but a click away, the question becomes not whether we can outdo the 20th century, but whether we can do so without physical war.

Fifty years ago, physicists, engineers, and mathematicians completed a fifty-year sprint that shipped the transistor, the laser, fiber optics, the communication satellite, the telephone, and the UNIX operating system, enabling the move from physical to virtual space. Bell Labs, a spinoff R\&D lab of AT\&T, and arguably the most successful science and technology project known to history, was able to do this not because its scientists were of different genetic makeup, but because they had access to healthy consensus-making machines, facilitating wide, cross-disciplinary collaboration, fueled in big part by the threat of Germany and the USSR.

In stark contrast, the 21st century starts out distracted. William Shockley, the inventor of the transistor, was born in 1910, did his doctorate at Caltech and was later recruited by the band of pirates that was Bell Labs. But had he been born at the point of his death in 1989, he would have either had his spirit crushed in SAT-prepped academia, perhaps sidetracked by the minutia of quantum gravity, or at best later been fired from Google for his views on race and intelligence. We seem to forget that science is not a body of knowledge, but a process fueled by errors, and that when the frontier is no longer the telephone, it is unwise to exile the very types of minds that could discover teleportation.

Further illustrating the decay of academia, there could be nothing more anachronistic than the opening sequence of the 2001 movie, A Beautiful Mind. The scene is set in 1947, and the protagonist, John Nash, is one of a dozen young mathematicians who have just arrived at Princeton University. The students, all clad in sharlpy tailored suits, are seated in the mathematics department lounge with tobacco smoke in the air and portraits of great thinkers gracing the dark-mahogany walls, receiving the dean's welcoming words:

\begin{quote}
Mathematicians won the war. Mathematicians broke the Japanese codes... and built the A-bomb. Mathematicians... like you. The stated goal of the Soviets is global Communism. In medicine or economics, in technology or space, battle lines are being drawn. To triumph, we need results. Publishable, applicable results. Now who among you will be the next Morse? The next Einstein? Who among you will be the vanguard of democracy, freedom, and discovery? Today, we bequeath America's future into your able hands. Welcome to Princeton, gentlemen.
\end{quote}

With government spineless, and academia tracked, today's talent finds itself drawn to more vital institutions like Silicon Valley (which in part provides a sociological explanation for the stagnation of the physical sciences). At one point the valley functioned as renegade haven, but now it is entering a similar stasis like Washington, DC, and Cambridge, MA. Big Tech formalized the Bell Labs model, but it has been living off of the same set of 50-year old ideas produced by research institutions that are now dead. And while it is possible to reform from the inside, it is at this point more likely that change will come from outside.

\newpage

\section*{Stuck in Commercialization}

If the founding of Bell Labs in 1925 marks the start of a period of research, and Apollo 11 landing on the moon in 1969 delineating a period of development, then the release of Apple II in 1977 marks the peak of Silicon Valley, and the transition to pure commercialization. If it is not to close out as a century defined by globalized rentseeking, the 21st century is in dire need of open research institutions, as technological progress is not just about inventing things, it is also about inventing ways of inventing things.

Google has famously gone above and beyond to copy the Bell Labs model. With its great cash reserves, it has built sprawling campuses and informal office environments aimed at fostering creativity. And they pay high multiples of the market rate for ostensibly the brightest minds. Google even has the 20 percent rule which encourages employees to spend 20 percent of their time on side projects not directly related to their main tasks in order to inspire exploration. The 20 percent rule was directly inspired by Bell Labs and has led to some of Google's successful products like Gmail and Google News. But as American mathematician Charles Peirce feared, "I am afraid there will be little tangible left in a later age, to remind our heirs that we were men, rather than cogs in a machine."\cite{ref1} Like the amateur cyclist with high-end gear, Big Tech has managed to copy Bell Labs in all the ways that do not matter.

Today's Big Tech engineer is of a different breed than the Bell engineer. Whilst certainly smart, he will not build mechanical maze-solving mice in his spare time, but instead read "The Lean Startup" and practice Leetcode. The Big Tech engineer views work not as honorable, but a continuation of school, a series of tasks to be done – a line item on the resume. Pirates are not bought with a nice boat – they pledge their allegiance to whoever is going out on a great adventure with the slight possibility of loot and glory. And because Big Tech lacks adventure, it must overcompensate with loot – naturally attracting maintainers, not pirates.

Another tempting contemporary Bell Labs analogue may be OpenAI. And while it is not yet as tracked as the rest of Big Tech, it is a closed project of narrower scope. The transistor was a multidisciplinary achievement involving solid state physics, material science, electrical engineering, chemistry, mechanical engineering, and mathematics. ChatGPT was built using all but one of those disciplines. This is not a competition in what project can invoke the highest number of disciplines. Nor am I questioning the utility of a product like ChatGPT – I am using it to write this very article. And OpenAI might one day create "safe AGI," but the company is not yet as exploding and cross-disciplined of an institution as Bell Labs was.

So where does one go to learn all these disciplines? Out of all of the STEM fields, computer science, a subfield of mathematics, is arguably the easiest to self-study, which in part explains why software has become such a popoular resort for the institutionless and benefited from an explosion in creativity thus. The colleges will deny the effectiveness of self-study no matter the field; that knowledge acquired in one's chamber is somehow inferior to knowledge acquired in a classroom setting. The reactionary view becomes that there is no value whatsoever to communal learning. But as with most extremist positions, they are a tell that both parties are overcompensating. If MOOCs were a sufficient substitute, the dropout would not feel the need to constantly assert it (but as a culture we are still overindexed on college, so I will not countersignal the dropout anymore). Likewise, if our institutions would have been working, they would not have been so occupied with trying to deny lone genius theory, or a priori discount the ideas from those who try their luck outside, it would have been a non-issue. But because the colleges are not delivering, they need to excommunicate critics, and belittle the alternatives.

But what are the alternatives? The most recent Bell-like institution is Apple. Apple was founded in 1976, and in its first 20 years it completely defined a new era of personal computing and technology business. Apple would not have been able to do this without Steve Jobs' vision. When Bell Labs was first developing the mobile phone, their marketing study in 1971 concluded that "there was no market for mobile phones at any price." Engel, the head of cellular system design at Bell, later came to believe that marketing studies could only tell you something about the demand for products that already exist.\cite{ref2} Steve Jobs had a similar view that people do not know what they want until you show it to them.

A more recent attempt is Y Combinator. YC is a San Francisco-based startup accelerator which has produced some of the highest market capitalizations of the 21st century, including Airbnb, Dropbox, and Stripe. Again we can pose the question whether these services have pushed civilization to the same degree as the transistor. YC claims to have been successfully able to formalize 1/3 of technology business which points at a fundamental issue: the YC adherent will outsource thinking whenever possible – to the crowd, rules, or computers – because fundamentally, he does not believe his brain. "Make something people want" is great business advice, but had Jobs and Wozniak followed it, they likely would have ended up a computer parts retailer. But in the end, even Apple was a project of commercialization. Apple did not make scientific breakthroughs, but rather took previous ideas and made them accessible to the masses. And this commercialization has only accelerated in the recent decade. Apple had a telos, but the modern west-coast entrepreneur will conduct a random walk until all old ideas are exhausted by siloed companies – in turn causing academic overspecialization. This leaves us with all but one contemporary institution: the internet.

\newpage

\section*{Heeding the Instinctive Call}

The problem with the internet is that it does not offer the same adventure as crossing the Atlantic for the first time, or racing against the Soviets to the atom bomb. In contrast to Bell Labs, the internet is not teleological. The facilitation of unrestricted flow of information is not rooted in a deep need like survival, or virtue like honor. This meaninglessness recently reflects itself in culture with the "Literally Me" meme.

The "Literally Me" meme portrays an archetype embodied by fictional characters such as Rust Cohle in True Detective, William Foster in Falling Down, and Tyler Durden in Fight Club. These characters are often nihilists whilst still being portrayed as being "right" in some sense. The archetype's name derives from the fact that so many viewers indeed feel like it is literally them. And it is certainly not your grandparents, but frustrated 20-plus guys, who feel a sense of absurdity and disappointment for the status quo. Young men used to be able to exercise their adventourous spirit in war or by sailing the seas, or satiate their intellectual hunger in academia. But now all the seas are explored, the wars are fake, and a degree is more degrading than it is intellectually stimulating. The litterally me phenomenon represents the frustration of having these natural instincts denied.

But we cannot escape up in the cloud for the simple reason that humans are embodied. Contrary to what many propagated during the remote-boom of 2021, ideation is stunted in virtual space, as humans need to interact in physical space to feel alive and use their fullest potential. And while it is true that the logic of violence has changed and the wars are now informational, TikTok is too abstract an enemy to be a comparable enemy to the USSR. So the internet alone is not enough.

The network state solves the internet's denial of nature. The network state, originally conceptualized by the author James Dale Davidson, and recently popularized by the technology entrepreneur Balaji Srinivasan, is about instantiating value-driven internet communities in physical space with the ultimate goal of diplomatic recognition. And new ideas are a surefire way of surpassing legacy nation's GDPs that are still resting on their laurels. But for the network state to become the replacement institution of the 21st century, its Raison d'être must not be political, but to enable the return to human nature – heeding the instinctive call to adventure.

The turn-around scientists of the 21st century will be dropouts and excommunicados driven by virtue. The current paradigm has become too dogmatic, and paradigm shifts require an individual that is green enough to have not been indoctrinated into it, but who is also not driven by novelty alone.

Honor is the instinctive sense in which Bell Labs feels anachronistic. The surface-level reasons are the old school dress code, and the frequent cigaratte breaks – both overlooked practices that contribute to a culture of seriousness and socialization. But the more vicious explanation is that honor is forgotten. Bell Labs presented the opportunity to be in walking distance of "the guy who wrote the book" on circuits, information theory, or electromagnetism (and in fact encouraged by management to never hesitate bothering him for help). Unlike the modern west-coast engineer, the Bell engineer was not driven so much by loot, as the glory of carrying the torch of western civilization. Bell Labs had didn not have to lie about its importance, or overcompansate for its lack thereof. But if not war, there is still honor.

Physical organization matters. Bell Lab's headquarter office was known for its long hallways, designed to encourage serendipitous interactions among researchers who would otherwise never meet: As Jon Gertner observed, "Physical proximity [to the guy who wrote the book] was everything. People had to be near one another. Phone calls alone wouldn't do."\cite{ref3} Today, "the guy who wrote the book" is in the nursing home. Furhermore, remote work has been a powerful globalizing force, but it has not unequivocally improved the quality of ideas – if anything it has narrowed the definition of "technology." Akin to how college students can only come up with business ideas for college students, a life lived through of apps can only think of such. Technology is converging towards the same level of sophistication as a laundrymat and the way out is blue-sky research. With the right population and constitution, a network state could provide timelines akin to federal funding, and motivation akin to war.

Technology is the weapon at our disposal. No matter the conspiratorial programs (whose existence we should not in principle rule out), there is always wiggle room for individual action. And with the network state as the replacement institution, technology is once again about to display its inherent political nature and disruptive power. “There is no need to fear or to hope, but only to look for new weapons.”\cite{ref4}

\begin{thebibliography}{9}
\bibitem{ref1}
Jon Gertner, \textit{The Idea Factory: Bell Labs and the Great Age of American Innovation} (Penguin Books, 2012), 360.
\bibitem{ref2}
Gertner, \textit{The Idea Factory: Bell Labs and the Great Age of American Innovation}, 289.
\bibitem{ref3}
Gertner, \textit{The Idea Factory: Bell Labs and the Great Age of American Innovation}, 151.
\bibitem{ref4}
Gilles Deleuze, \textit{Postscript on the Societies of Control} (The Anarchist Library, 1992), 3.
\end{thebibliography}

\end{document}
